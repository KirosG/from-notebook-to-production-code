\documentclass[12pt,t]{beamer}
\usetheme{Rochester}
\usepackage{pythonhighlight}
\usepackage{graphicx}
\usepackage{tikz}
\usepackage{amsmath}
\usecolortheme{whale}
\newcommand{\bi}{\begin{itemize}}                                 % begin itemize
\newcommand{\ei}{\end{itemize}}                                   % end itemize
\newcommand{\be}{\begin{enumerate}}                               % begin itemize
\newcommand{\ee}{\end{enumerate}}                                 % end itemize
\newcommand*{\ig}[2]{{\centering\includegraphics[#1]{#2}\par}}    % include graphics
\newcommand{\txt}{\texttt}										  % texttt

\setbeamercolor{normal text}{fg=black}
\setbeamerfont{normal text}{size=\normalsize}
\setbeamercolor{title text}{fg=black}
\setbeamerfont{frametitle}{size=\Large}
\setbeamercolor{section in toc}{fg=black}

\title[FNTPC]{From Notebook to Production Code}
%\subtitle{Numbers Handwritten by a Computer}
\author[Messier]{Christopher Messier}
\institute[DSI-DC]{General Assembly\\Washington, DC}
%\date{\today}}

\begin{document}
\linespread{1.5}

\begin{frame}
	\titlepage
\end{frame}

%%% Outline

%%%%%%%%%%%%%%%%%%%%%%%
\section{Introduction}

\begin{frame}{Overview}
	\tableofcontents
\end{frame}

\begin{frame}{What's wrong with a Notebook?}
	An important tool, but have disadvantages
	\begin{itemize}
		\item Non-executable
		\item No returns
		\item JavaScript Bloat (\emph{ewwww})
		\item \emph{slow}
	\end{itemize}
\end{frame}

\begin{frame}{What won't be covered}
	There's limited time available so we'll be skipping over...
    \begin{itemize}
	    \item Optimization
	    \item Unit Testing
		\item Packaging
		\item So much more...
	\end{itemize}
\end{frame}

\begin{frame}{Getting Started}
	We'll be working with the notebook found here:
	\small{\texttt{github.com/messiest/from-notebook-to-production-code}}
	\begin{itemize}
		\item What it is:
		\begin{itemize}
			\item A quick introduction to \texttt{PyTorch}
			\item Implementation of the LeNet
			\item Convolutional Neural Network on \emph{MNIST}
		\end{itemize}
	\end{itemize}
\end{frame}

\section{Code Conversion}

\begin{frame}{Converting Your Code}
	Exporting your code to a Python (\texttt{.py}) file
	\begin{itemize}
		\item File $\to$ Download As $\to$ Python (.py)
		\item Move file from Downloads to the Project directory
		\item Clean out notebook ''artifacts''
	\end{itemize}
\end{frame}

\section{Build File}

\begin{frame}{Build File}
	Install all packages required for a project
	\begin{itemize}
		\item \texttt{requirements.txt} - list of all packages
		\item Could be either a Python (\texttt{.py}) or Bash (\texttt{.sh}) script
		\item Ensures that there is a common build for all users
	\end{itemize}
\end{frame}

\section{Command Line Interface}

\begin{frame}{Command Line Interface}
	Interacting with your code through the command line
	\begin{itemize}
		\item Makes your code more flexible
		\item Gives greater control over code behavior
		\item Provides a common interface for 
	\end{itemize}
\end{frame}

\begin{frame}{\texttt{argparse}}
	Part of the Python Standard Library (included w/ Python)
	\begin{itemize}
		\item Allows you to define custom arguments
		\item Provides help read-outs by default
		\item Easy to use, and descriptive
	\end{itemize}
\end{frame}

\section{Progress Bar}

\begin{frame}{Progress Bar}
	Provides a simple way to track your program's progress
	\begin{itemize}
		\item Estimate how long your program will run
		\item Track values of important variables
		\item Provides a professional look/feel
	\end{itemize}
\end{frame}

\begin{frame}{\texttt{tqdm}}
	Provides a wrapper to output progress to command line
	\begin{itemize}
		\item Wraps an iterable, and automatically writes output
		\item Provides a way to customize the progress bar
	\end{itemize}
\end{frame}

\begin{frame}[fragile, c]  % image frame
	\begin{center}
		\Large{Thank You}
	\end{center}
\end{frame}

\end{document}