\documentclass[12pt,t]{beamer}
\usetheme{Rochester}
\usepackage{pythonhighlight}
\usepackage{hyperref}
\usepackage{graphicx}
\usepackage{tikz}
\usepackage[normalem]{ulem}
\usepackage{amsmath}
\usecolortheme{whale}
\newcommand{\bi}{\begin{itemize}}                                 % begin itemize
\newcommand{\ei}{\end{itemize}}                                   % end itemize
\newcommand{\be}{\begin{enumerate}}                               % begin itemize
\newcommand{\ee}{\end{enumerate}}                                 % end itemize
\newcommand*{\ig}[2]{{\centering\includegraphics[#1]{#2}\par}}    % include graphics
\newcommand{\txt}{\texttt}										  % texttt

\setbeamercolor{normal text}{fg=black}
\setbeamerfont{normal text}{size=\normalsize}
\setbeamercolor{title text}{fg=black}
\setbeamerfont{frametitle}{size=\Large}
\setbeamercolor{section in toc}{fg=black}

\title[FNTPC]{From Notebook to Production Code}
\author[Messier]{Christopher Messier}
\institute[DSI-DC]{General Assembly\\Washington, DC}
%\date{\today}}

\begin{document}
\linespread{1.5}

\begin{frame}
	\titlepage
\end{frame}

%%% Outline

%%%%%%%%%%%%%%%%%%%%%%%
\section{Introduction}

\begin{frame}{Overview}
	\tableofcontents
\end{frame}

\begin{frame}{About Me (Chris Messier)}
	\begin{itemize}
		\item Data Scientist at MainStreet Bank
		\item BAH Data Science Fundamentals IA
		\item Background in academic Economics
		\item Research Interests
		\begin{itemize}
			\item NLP
			\item Computer Vision
			\item Probabilistic Programming
			\item Graphical Models
		\end{itemize}
	\end{itemize}
\end{frame}

\begin{frame}{What Does a Data Scientist \emph{Produce}?}
	\begin{itemize}
		\item Analytics
		\begin{itemize}
			\item Explaining the data and giving recommendations
		\end{itemize}
		\pause
		\item Research
		\begin{itemize}
			\item Develop a deep understanding of a topic given the data
		\end{itemize}
		\pause
		\item Solutions
		\begin{itemize}
			\item Developing software that enables data-driven solutions
		\end{itemize}
	\end{itemize}
\end{frame}

\begin{frame}{What Is Production Code?}
	\emph{Production Code} is...
	\begin{itemize}
		\item Code used in a software product
	\end{itemize}
\end{frame}

\begin{frame}{Why Worry About Production Code?}
	Production
\end{frame}

\begin{frame}{What will be covered}
	\begin{itemize}
		\item Using the commandline
		\begin{itemize}
			\item bash profiles/scripting
		\end{itemize}
		\item Converting Code
		\begin{itemize}
			\item Notebook $\to$ Python files
		\end{itemize}
		\item Best Practices
		\begin{itemize}
			\item Project structure
			\item Code style
		\end{itemize}
	\end{itemize}
\end{frame}

\begin{frame}{What won't be covered}
	There's limited time available so we'll be skipping over...
    \begin{itemize}
	    \item Optimization
	    \item Unit Testing
		\item Packaging
		\item So much more...
	\end{itemize}
\end{frame}

\begin{frame}{Why the preference for Mac?}
	\begin{itemize}
		\item \sout{Aesthetics?}
		\item \sout{Hardware?}
		\item UNIX operating system
	\end{itemize}
\end{frame}

\begin{frame}{UNIX}
	\begin{itemize}
		\item A (large) family of multi-task operating systems
		\item Started development in 1969 at Bell Labs
		\item The foundation for all Linux flavors
		\item macOS is the most installed UNIX base
	\end{itemize}
\end{frame}

\begin{frame}{UNIX \emph{cont.}}
	Linux is the choice of most production workstations/servers. What does this mean?
	\begin{itemize}
		\item wide user/support base
		\item similar file systems for development/deployment
		\item similar backends/core libraries
		\item minimal changes between dev/deploy
		\item common interfaces (\emph{bash})
	\end{itemize}
\end{frame}

\begin{frame}{bash}
	A UNIX shell command language
	\begin{itemize}
		\item Default login shell for macOS and most Linux
		\item Defines the interactions with the command line
		\item Enables you to...
		\begin{itemize}
			\item Navigate the file system
			\item Launch programs
		\end{itemize}
	\end{itemize}
\end{frame}

\begin{frame}{bash profile}
	A way to customize your bash environment
	\begin{itemize}
		\item customize aesthetics
		\item create shortcuts
	\end{itemize}
	Your bash profile can be found at:
	\begin{itemize}
		\item \url{~/.bash_profile}  % url is convenient for file paths as well
	\end{itemize}
\end{frame}

\begin{frame}{Getting Started}
	We'll be working with the project here:
	\small{\url{www.github.com/messiest/pytorch-image-classifiers}}
	\begin{itemize}
		\item What it is:
		\begin{itemize}
			\item An implementation of the LeNet on \emph{MNIST} data
			\item An implementation of the GoogLeNet on \emph{CIFAR10} data
		\end{itemize}
	\end{itemize}
\end{frame}

\begin{frame}{Accessing the Example Code}
	The code is already included as a submodule to this repo.
	To populate the \texttt{pytorch-image-classifiers/} directory use the following commands:
	\begin{itemize}
		\item \texttt{git init submodule}
		\item \texttt{git submodule update}
	\end{itemize}
\end{frame}

\section{Code Conversion}

\begin{frame}{What's wrong with a Notebook?}
	An important tool, but have disadvantages
	\begin{itemize}
		\item Non-executable
		\item No returns
		\item JavaScript Bloat (\emph{ewwww})
		\item "\emph{slow}"
	\end{itemize}
\end{frame}

\begin{frame}{Converting Your Code}
	Exporting your code to a Python (\texttt{.py}) file
	\begin{itemize}
		\item File $\to$ Download As $\to$ Python (.py)
		\item Move file from Downloads to the Project directory
		\item Clean out notebook ''artifacts''
	\end{itemize}
\end{frame}

\section{Build File}

\begin{frame}{Build File}
	Install all packages required for a project
	\begin{itemize}
		\item \texttt{requirements.txt} - list of all packages
		\item Could be either a Python (\texttt{.py}) or Bash (\texttt{.sh}) script
		\item Ensures that there is a common environment for all users
		\item Good time to use a \emph{virtual environment}
	\end{itemize}
\end{frame}

\section{Command Line Interfaces}

\begin{frame}{Command Line Interface}
	Interacting with your code through the command line
	\begin{itemize}
		\item Makes your code more flexible
		\item Gives greater control over code behavior
		\item Provides a common interface for 
	\end{itemize}
\end{frame}

\begin{frame}{\texttt{argparse}}
	Part of the Python Standard Library (included w/ Python)
	\begin{itemize}
		\item Allows you to define custom arguments
		\item Provides help read-outs by default
		\item Easy to use, and descriptive
	\end{itemize}
\end{frame}

\section{Bonus Features}

\begin{frame}{Progress Bar}
	Provides a simple way to track your program's progress
	\begin{itemize}
		\item Estimate how long your program will run
		\item Track values of important variables
		\item Provides a professional look/feel
	\end{itemize}
\end{frame}

\begin{frame}{\texttt{tqdm}}
	Provides a wrapper to output progress to command line
	\begin{itemize}
		\item Wraps an iterable, and automatically writes output
		\item Provides a way to customize the progress bar
	\end{itemize}
\end{frame}

\begin{frame}[fragile, c]  % image frame
	\begin{center}
		\Large{Thank You}
	\end{center}
\end{frame}

\end{document}